\section{Environment}\label{sec:environment}
All major components of Flatland are handled by the environment.  As such, it is important to become familiar with some of the fundamental aspects in order to fully understand how the environment functions in Flatland. \\

\noindent Key components:

\begin{itemize}
	\item Classes
	\begin{itemize}
		\item Grid Transition Map
		\item Agent
	\end{itemize}
	\item Builders
	\begin{itemize}
		\item Observation Builder
		\item Malfunction Builder
	\end{itemize}
	\item Generators
	\begin{itemize}
		\item \textbf{Rail Generator}: generates rail network
		\item \textbf{Line Generator}: generates start and end locations for each agent
		\item \textbf{Timetable Generator}: generates earliest departure and latest arrival times for each agent
	\end{itemize}
	\item Functions
	\begin{itemize}
		\item \texttt{reset(self)} function
		\item \texttt{step(self, action\_dict)} function: runs a single step in the simulation of the environment
	\end{itemize}
	\item \texttt{action dictionary}: actions corresponding to agents at a given time step
\end{itemize}

\subsection{What is the environment and its properties?}
Text.

\subsection{What is the output of the environment?}
Text.

\subsection{What does the environment require for a solution?}
Text.
