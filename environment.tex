\section{Environment}\label{sec:environment}

\subsection{What is the environment and its properties?}
In a nutshell, the environment is responsible for organizing all aspects of Flatland that are necessary for simulating the movement of trains.  
There are three major responsibilities:
\begin{itemize}
	\item Gathering the primary underlying modules in a single location
	\item Providing the ability to perform a discrete-time simulation
	\item Rendering the output of a simulation
\end{itemize}

\subsubsection{Underlying modules}
By gathering the underlying modules, Flatland can virtually represent the key components of a railway network.  In particular, there are infrastucture components and operation components.
\textbf{Infrastucture components} include the tracks and the trains.  These are supported by the \textit{Grid Transition Map}, the \textit{Rail Generator}, and the \textit{Agent}.
\textbf{Operation components} include the schedules, paths, and delays.  These are supported by the \textit{Timetable Generator}, the \textit{Line Generator}, and the \textit{Malfunction Builder}.

\subsection{What is the output of the environment?}
There are three ? when considering the output of the environment:
\begin{enumerate}
	\item Intended output
	\item Intermedieate output
	\item Accessible attributes
\end{enumerate}

\subsection{What does the environment require for a solution?}
Text.
