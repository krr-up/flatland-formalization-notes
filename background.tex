\section{Background}\label{sec:background}

The environment organizes all aspects of Flatland necessary for simulating the movement of trains.  
Understanding components that are accessible via the environment is key to fully understanding the environment itself: \\

\begin{table}[H]
\centering
	{\footnotesize
  	\begin{tabular}{@{\hspace*{1em}}ll@{}}
    	\toprule \toprule
    	\multicolumn{2}{c}{Key components of Flatland} \\

    	\midrule

    	\textbf{Classes} \\
    	\tabitem Grid Transition Map & defines the tiles and transitions of the grid \\
    	\tabitem Agent & defines the trains that traverse the environment \\
    	\tabitem Observation Builder & defines what agents know about the environment \\[.75\normalbaselineskip]
    
	\textbf{Generators} \\
    	\tabitem Rail Generator & generates rail network \\
	\tabitem Line Generator & generates start and end locations for each agent \\
    	\tabitem Timetable Generator & generates departure and arrival times for each agent \\
	\tabitem Malfunction Generator & generates malfunctions that simulate delays \\[.75\normalbaselineskip]
    
	\textbf{Functions} \\
    	%\tabitem \texttt{reset(self)} & resets the environment to its initial state \\
	\tabitem \texttt{step(self, action\_dict)} & runs a single step in the simulation of the environment \\
	 & \texttt{action\_dict} \textit{contains one action for each agent} \\

	\bottomrule
  	\end{tabular}
	}
\end{table}

%\noindent These components reveal information about Flatland that is accessible via the environment, such as train location, expected arrivals, or current malfunctions.
